\chapterquote{Calculus is the most powerful weapon of thought yet devised by the wit of man.}{Wallace B. Smith}

The proofs in this section assume a basic understanding of calculus. It's fine if you don't; you can just ignore the proofs. But I believe 3D geometry is most naturally understood with calculus, and it is one of the most natural and elegant exercises of calculus.

\section{Definitions}

Before we \textit{prove} that the formulas in all their glory, we need to define volume and surface area.

\subsection{Volume}

\begin{defi}[Volume]
The volume of a solid is the integral of the areas of all cross-sections made with planes parallel to a certain reference plane.
\end{defi}

This basically means you sum up all of the cross-sections, similar to finding the area of a triangle. Usually our reference plane will be the base.

\begin{fact}[Cavalieri's Principle]
If in two solids of equal altitudes, the planes parallel to and at the same distance from their respective bases always create cross-sections with equal area, then the two solids have the same volume.
\end{fact}

\begin{center}
    \begin{tikzpicture}
    \newcommand{\size}{1.5}
    \draw (-\size,0) arc (180:360:{\size} and 0.6);
    \draw[dashed] (\size,0) arc (0:180:{\size} and 0.6);
    \draw (-\size,\size+\size) arc (180:360:{\size} and 0.6);
    \draw (\size,\size+\size) arc (0:180:{\size} and 0.6);
    \draw (-\size,0)--(-\size,\size+\size);
    \draw (\size,0)--(\size,\size+\size);
    %\draw[dashed] (\size,0)--(0,0)--(0,\size+\size);
    %\node at (0,-\size+\size/4) {\textit{A right cylinder.}};
    %\filldraw (0,0) circle (1pt);
    %\node at (\size/2,\size/8) {$r$};
    %\node at (\size/6,\size) {$h$};
    \end{tikzpicture}
    \hspace{1.5cm}
    %\textit{A right cylinder.}
    %\\[1\baselineskip]
    \begin{tikzpicture}
    \newcommand{\size}{1.5}
    \newcommand{\slant}{0.5}
    \draw (-\size,0) arc (180:360:{\size} and 0.6);
    \draw[dashed] (\size,0) arc (0:180:{\size} and 0.6);
    \draw (-\size+\slant,\size+\size) arc (180:360:{\size} and 0.6);
    \draw (\size+\slant,\size+\size) arc (0:180:{\size} and 0.6);
    \draw (-\size,0)--(-\size+\slant,\size+\size);
    \draw (\size,0)--(\size+\slant,\size+\size);
    %\draw[dashed] (\size,0)--(0,0);
    %\draw[dashed] (\size+\slant,0)--(\size+\slant,\size+\size);
    %\node at (0,-\size+\size/4) {\textit{An oblique cylinder.}};
    %\filldraw (0,0) circle (1pt);
    %\node at (\size/2,\size/8) {$r$};
    %\node at (\slant+\size+\size/6,\size/5+\size/5+\size/5+\size/5) {$h$};
    \end{tikzpicture}
\end{center}

\begin{fact}[Rearranging Principle]
If you can rearrange a solid into another solid, the two solids have the same volume.
\end{fact}

This means that if you have a rectangular prism attached to another, you can just find the sum of the volumes of the two rectangular prisms to find the volume of the joined solid. Similar methods apply for other combinations of solids.

\subsection{Surface Area}
Surface area is surprisingly hard to define formally. Intuitively, it is just the area of the surfaces.

\begin{fact}[Additivity]
The surface area of an object is the sum of the surface area of its parts.
\end{fact}

\begin{fact}[Surface Area of Flat Shapes]
The surface area of a flat shape is the same as the area of the flat shape.
\end{fact}

\begin{fact}[Straight Lines]
If part of the surface consists of lines, then the surface area of that part can be found by integrating the lengths of the lines.
\end{fact}
As an example, consider the side of a cylinder or the lines joining the apex of a cone to the circumference of its base.

\begin{fact}[Curves]
If part of the surface consists of curves, then the surface area of that part can be found by integrating the lengths of the curves.
\end{fact}
This will be useful for spheres.

\section{Prisms and Cylinders}

Prisms and cylinders are similar in that a cylinder is analogous to a prism with a continuous base. We also discuss some specific prisms, such as cubes, rectangular prisms, and parallelepipeds.

\subsection{Prisms and Cylinders}
We define a prism and a cylinder.

\begin{defi}[Prism]
A prism is a solid with two congruent parallel bases with parallelograms for the side faces.
\end{defi}

An equivalent definition is that a prism is a solid with two bases that can be translated to each other.

\begin{defi}[Cylinder]
A cylinder is a solid with two parallel circles as bases.
\end{defi}

\begin{defi}[Right Prism and Cylinder]
A prism or cylinder is considered \textit{right} if a line joining two corresponding points on the two bases is perpendicular to both bases.
\end{defi}

Unless otherwise specified, prisms and cylinders are right. Regardless, the volume formulas \textit{always} hold.

\begin{center}
    \begin{tikzpicture}
    \newcommand{\size}{1.5}
    \draw (-\size,0) arc (180:360:{\size} and 0.6);
    \draw[dashed] (\size,0) arc (0:180:{\size} and 0.6);
    \draw (-\size,\size+\size) arc (180:360:{\size} and 0.6);
    \draw (\size,\size+\size) arc (0:180:{\size} and 0.6);
    \draw (-\size,0)--(-\size,\size+\size);
    \draw (\size,0)--(\size,\size+\size);
    \draw[dashed] (\size,0)--(0,0)--(0,\size+\size);
    \node at (0,-\size+\size/4) {\textit{A right cylinder.}};
    %\filldraw (0,0) circle (1pt);
    \node at (\size/2,\size/8) {$r$};
    \node at (\size/6,\size) {$h$};
    \end{tikzpicture}
    \hspace{1.5cm}
    %\textit{A right cylinder.}
    %\\[1\baselineskip]
    \begin{tikzpicture}
    \newcommand{\size}{1.5}
    \newcommand{\slant}{0.5}
    \draw (-\size,0) arc (180:360:{\size} and 0.6);
    \draw[dashed] (\size,0) arc (0:180:{\size} and 0.6);
    \draw (-\size+\slant,\size+\size) arc (180:360:{\size} and 0.6);
    \draw (\size+\slant,\size+\size) arc (0:180:{\size} and 0.6);
    \draw (-\size,0)--(-\size+\slant,\size+\size);
    \draw (\size,0)--(\size+\slant,\size+\size);
    \draw[dashed] (\size,0)--(0,0);
    \draw[dashed] (\size+\slant,0)--(\size+\slant,\size+\size);
    \node at (0,-\size+\size/4) {\textit{An oblique cylinder.}};
    %\filldraw (0,0) circle (1pt);
    \node at (\size/2,\size/8) {$r$};
    \node at (\slant+\size+\size/6,\size/5+\size/5+\size/5+\size/5) {$h$};
    \end{tikzpicture}
\end{center}

\begin{theo}[Volume of a Prism]
The volume of a prism with a base of area $B$ and a height of $h$ is $Bh.$
\end{theo}

\begin{theo}[Volume of a Cylinder]
The volume of a cylinder with a base if area $B$ and a height of $h$ is $Bh.$
\end{theo}
An alternative formulation of this is that the volume of a cylinder with radius $r$ and height $h$ is $\pi r^2h.$

We prove both volume formulas in one fell swoop.

\begin{pro}
Let the reference plane be one of the bases. Then note that the cross-section always has area $B$ over a height of $h.$ Let $k$ be the distance of the cross-section from the base. Then the volume is $\int\limits_{0}^h Bkdk=Bh.$
\end{pro}

\begin{theo}[Surface Area of a Right Cylinder]
The surface area of a right cylinder with radius $r$ and height $h$ is $2\pi r^2+2\pi rh.$
\end{theo}

\begin{pro}
The two faces have area $\pi r^2$ each, and the lateral surface area (surface area of the sides) can be found by integrating the circumference of the base about the height. So the lateral surface area is $\int\limits_0^h 2\pi rdx=2\pi rh.$ Thus the total surface area is $2\pi r^2+2\pi rh.$
\end{pro}

\subsection{Cubes and Rectangular Prisms}

We discuss cubes and rectangular prisms.

\begin{defi}[Cube]
A cube is a solid with $6$ square faces.
\begin{center}
% \usetikzlibrary{positioning,calc}
\begin{tikzpicture}
            \newcommand{\side}{3}
            \newcommand{\otherside}{1.3}
            \draw[dashed] (\otherside,\side+\otherside) -- (\otherside,\otherside) -- (\side+\otherside,\otherside)
            (0,0) -- (\otherside,\otherside);
            \draw (0,0) rectangle (\side,\side)
            (0,\side) -- (\otherside,\otherside+\side) -- (\otherside+\side,\otherside+\side) -- (\side,\side)
            (\otherside+\side,\otherside+\side) -- (\otherside+\side,\otherside) -- (\side,0);
\end{tikzpicture}
\end{center}
\end{defi}

\begin{theo}[Volume of a Cube]
The volume of a cube with side length $x$ is $x^3.$
\end{theo}

\begin{pro}
A cube is a prism with base area $x^2$ and height $x,$ so its area is $x^3.$
\end{pro}

\begin{theo}[Surface Area of a Cube]
The surface area of a cube with side length $x$ is $6x^2.$
\end{theo}

\begin{pro}
There are six faces, and each face has surface area $x^2.$ Thus the total surface area is $6x^2.$
\end{pro}

\begin{defi}[Rectangular Prism]
A rectangular prism is a prism with rectangular bases.
\begin{center}
% \usetikzlibrary{positioning,calc}
\begin{tikzpicture}
            \newcommand{\wide}{3}
            \newcommand{\figlength}{4}
            \newcommand{\high}{2}
            \newcommand{\otherside}{1.3}
            \draw[dashed] (\otherside,\high+\otherside) -- (\otherside,\otherside) -- (\figlength+\otherside,\otherside)
            (0,0) -- (\otherside,\otherside);
            \draw (0,0) rectangle (\figlength,\high)
            (0,\high) -- (\otherside,\otherside+\high) -- (\otherside+\figlength,\otherside+\high) -- (\figlength,\high)
            (\otherside+\figlength,\otherside+\high) -- (\otherside+\figlength,\otherside) -- (\figlength,0);
\end{tikzpicture}
\end{center}
\end{defi}

In this chapter, a rectangular prism will always refer to a \textit{right} rectangular prism. Unless otherwise specified, rectangular prisms are right. This is generally true in competitions as well.

\begin{theo}[Volume of a Right Rectangular Prism]
The volume of a rectangular prism with side lengths $l,w,h$ is $lwh.$
\end{theo}

\begin{pro}
The base has area $lw$ and the height is $h,$ so the volume is $lwh.$
\end{pro}

\begin{theo}[Surface Area of a Right Rectangular Prism]
The surface area of a rectangular prism with side lengths $l,w,h$ is $2(lw+wh+hl).$
\end{theo}

\begin{pro}
There are two faces with the dimensions $l\times w,$ $w\times h,$ and $h\times l.$ Multiplication and addition finish.
\end{pro}

\begin{exam}
Three faces of a rectangular prism have areas $6,10,15.$ Find the volume of the rectangular prism.
\end{exam}
\begin{sol}
Let the side lengths be $a,b,c.$ Note that
\[ab=6\]
\[bc=10\]
\[ca=15,\]
and the volume of the prism is $abc.$ Multiplying all of the expressions together gives us $(abc)^2=900,$ or $abc=30.$
\end{sol}

\subsection{Parallelepipeds}

\begin{defi}[Parallelepipeds]
A parallelpiped is a solid with $6$ parallelogram faces.
\begin{center}
    \begin{tikzpicture}[scale=0.7]
\draw (0,0) -- (4,0) -- (6.5,2) -- (7.5,5) -- (3.5,5) -- (1,3) -- (5,3) -- (7.5,5);
\draw (0,0) -- (1,3);
\draw (4,0) -- (5,3);
\draw [dotted] (0,0) -- (2.5,2) -- (3.5,5);
\draw [dotted] (2.5,2)--(6.5,2);
\draw [dotted] (5,3)--(5,0);
\node at (5.3,1.7) {$h$};
\end{tikzpicture}
\end{center}
\end{defi}
Note that all parallelepipeds are prisms.

\begin{theo}[Volume of a Parallelepiped]
The volume of a parallelepiped with a base of area $B$ and height $h$ is $Bh.$
\end{theo}
The surface area of a parallelepiped should also not be too hard to compute, though there is no nice general formula.

\section{Pyramids and Cones}
Pyramids and cones are similar in that a cylinder is analogous to a prism with a continuous base.

\subsection{Pyramids}

\begin{defi}
A pyramid is a solid with a polygonal base whose vertices are all joined to a point not in the plane of the base. This point is called the \textit{apex}.
\begin{center}
\begin{tikzpicture}[scale=0.8]
\draw (0,0)--(4,0)--(4.5,1.5)--(2,3.46410162)--cycle;
\draw (2,3.46410162)--(4,0);
\draw[dotted] (0,0)--(4.5,1.5);
\end{tikzpicture}
\end{center}
\end{defi}

\begin{theo}[Volume of a Pyramid]
The volume of a pyramid with a base of area $B$ and a height of $h$ is $\frac{Bh}{3}.$
\end{theo}

\begin{pro}
Let the reference plane be the plane through the apex parallel to the base and let $k$ be the distance of the cross-section from the reference plane. (The cross-section lies on the same side of the reference plane as the base.)

Then by similarity, the volume is $\int\limits_0^k B\frac{k^2}{h^2}dk=\frac{B}{h^2}\int\limits_0^k k^2dk=\frac{B}{h^2}\cdot\frac{h^3}{3}=\frac{Bh}{3}.$
\end{pro}

\subsection{Cones}

\begin{defi}
A cone is a solid with a circular base where every point on the base is joined to a point not in the plane of the base. This point is called the \textit{apex}.
\end{defi}

The volume formula is identical to the volume of a pyramid.

\begin{theo}[Volume of a Cone]
The volume of a cone with a base of area $B$ and a height of $h$ is $\frac{Bh}{3}.$ An alternative formulation of this is that the volume of a cone with radius $r$ and height $h$ is $\frac{\pi r^2h}{3}.$
\end{theo}

\begin{pro}
Let the reference plane be the plane through the apex parallel to the base and let $k$ be the distance of the cross-section from the reference plane. (The cross-section lies on the same side of the reference plane as the base.)

Then by similarity, the volume is $\int\limits_0^k B\frac{k^2}{h^2}dk=\frac{B}{h^2}\int\limits_0^k k^2dk=\frac{B}{h^2}\cdot\frac{h^3}{3}=\frac{Bh}{3}.$

To prove the alternate formulation, note that $B=\pi r^2.$
\end{pro}

\begin{defi}[Right and Oblique Cones]
A cone is right if the line joining the center of it bases with the apex is perpendicular to the base, and oblique otherwise.
\end{defi}
Unless otherwise specified, cones are right. This is generally true in competitions as well. Regardless, the volume formula \textit{always} holds.

\begin{center}
    \begin{tikzpicture}
    \newcommand{\size}{1.5}
    \draw (-\size,0) arc (180:360:{\size} and 0.6);
    \draw[dashed] (\size,0) arc (5:180:{\size} and 0.6);
    
    \draw (-\size,0)--(0,\size+\size)--(\size,0);
    \draw[dashed] (\size,0)--(0,0)--(0,\size+\size);
    
    \node at (\size/2,\size/8) {$r$};
    \node at (\size/6,\size-\size/6) {$h$};
    
    \node at (0,-\size+\size/4) {\textit{A right cone.}};
    \end{tikzpicture}
    \hspace{1.5cm}
    \begin{tikzpicture}
    \newcommand{\size}{1.5}
    \newcommand{\slant}{2}
    \draw (-\size,0) arc (180:360:{\size} and 0.6);
    \draw[dashed] (\size,0) arc (10:180:{\size} and 0.6);
    
    \draw (-\size,0)--(\slant,\size+\size)--(\size,0);
    \draw[dashed] (\size,0)--(0,0);
    \draw[dashed] (\slant,\size+\size)--(\slant,0);
    
    \node at (\size/2,\size/8) {$r$};
    \node at (\slant+\size/6,\size-\size/6) {$h$};
    
    \node at (0,-\size+\size/4) {\textit{An oblique cone.}};
    \end{tikzpicture}
\end{center}

\begin{theo}[Surface Area of a Right Cone]
The surface area of a cone with radius $r$ and height $h$ is $\pi r^2+2\pi r\sqrt{r^2+h^2}.$
\end{theo}

\begin{pro}
The area of the base is $\pi r^2,$ and the lateral surface area can be found by integrating the slant height over the circumference. The lateral surface area is $\int\limits_{0}^{2\pi r}\sqrt{r^2+h^2}dx=2\pi r\sqrt{r^2+h^2.}$ (Note that the slant height is $\sqrt{r^2+h^2}$ by the Pythagorean Theorem.) Adding gives $\pi r^2+2\pi r\sqrt{r^2+h^2}.$
\end{pro}

The surface area of an oblique cone is surprisingly hard to find.

\section{Spheres}

Spheres are the $3$ dimensional version of circles.
\begin{defi}[Sphere]
A sphere is the locus of points in space equidistant from a certain point. This point is called the center.
\begin{center}
    \begin{tikzpicture}
%  \shade[ball color = gray!40, opacity = 0.4] (0,0) circle (2cm);
  \draw (0,0) circle (2cm);
  \draw (-2,0) arc (180:360:2 and 0.6);
  \draw[dashed] (2,0) arc (0:180:2 and 0.6);
%  \fill[fill=black] (0,0) circle (1pt);
  \draw[dashed] (0,0 ) -- node[above]{$r$} (2,0);
\end{tikzpicture}
\end{center}
\end{defi}

\begin{theo}[Volume of a Sphere]
The volume of a sphere with radius $r$ is $\frac{4\pi r^3}{3}.$
\end{theo}

\begin{pro}
Instead we prove the volume of a hemisphere is $\frac{2\pi r^3}{3}.$ Let the reference plane be the base of the hemisphere.

Then let the cross-section have a distance of $k$ from the base. Then by the Pythagorean Theorem, the radius of the cross-section is $\sqrt{r^2-k^2},$ so the area of the cross-section is $\pi(r^2-k^2).$ Thus the volume is \[\int\limits_{0}^{r}\pi(r^2-k^2)dk=\pi r^3-\pi \int\limits_{0}^{r}k^2dk=\pi r^3-\frac{\pi r^3}{3}=\frac{2\pi r^3}{3}.\]
Multiplying by $2$ implies that the volume of the sphere is $\frac{4\pi r^3}{3}.$
\end{pro}

\begin{theo}[Surface Area of a Sphere]
The surface area of a sphere with radius $r$ is $4\pi r^2.$
\end{theo}

To prove this, we first make an observation of what the surface area of the sphere is.

\begin{fact}[Surface Area of a Sphere]
The surface area of a sphere is the integral of the circumferences of all cross-sections made with planes parallel to a certain reference plane.
\end{fact}

Now we can explicitly integrate.

\begin{pro} % THIS IS WRONG: FIX PLS
We instead prove that the surface area of a hemisphere, not counting the base, is $2\pi r^2.$

We integrate about the arc of the circumference. Let $\theta$ be the angle a point on the cross-section forms with the radius containing the foot from the point onto the base. Then we integrate about $t=r\theta.$ Note integrating the circumferences gives
\[\int_0^{\frac{\pi r}{2}}2\pi r\cos\frac{t}{r}dt=2\pi r^2.\]
Multiplying by $2$ implies that the surface area of the sphere is $4\pi r^2.$

\begin{center}
    \begin{tikzpicture}
%  \shade[ball color = gray!40, opacity = 0.4] (0,0) circle (2cm);
  \draw (2,0) arc (0:180:2);
  \draw [blue,dashed](-1.41421356,1.41421356) arc (180:360:1.41421356 and 0.4);
  \draw (-2,0) arc (180:360:2 and 0.6);
\draw[dashed] (2,0) arc (0:180:2 and 0.6);
\draw[blue,dashed] (0,1.41421356)--(0,0)--(1.41421356,1.41421356)--cycle;
\draw[dashed] (0,0)--(2,0);
\draw (0.3,0) arc (0:45:0.3);
\node at (0.5,0.2) {$\theta$};
%  \fill[fill=black] (0,0) circle (1pt);
%  \draw[dashed] (0,0 ) -- node[above]{$r$} (2,0);
\end{tikzpicture}
\end{center}
\end{pro}

\section{Cross-sections}

\begin{defi}[Cross-section]
A cross-section of a solid is the intersection of the solid with a plane.
\end{defi}

The word “solid” implies that the interior of the object is part of it.

\begin{theo}[Cross-section of a Sphere]
The cross-section of a sphere is a circle.
\end{theo}

You can also take cross-sections of a cube and a cone. The former is used sometimes in math competitions, and the latter produces a shape known as a \textit{conic}.

\begin{theo}[Cross-section of a Cube]
The cross-section of a cube can be a triangle, quadrilateral, pentagon, or hexagon.
\end{theo}

The heuristical reason this is true is because a cube has $6$ faces, and the plane can intersect the cube at any $6$ of those faces. It's not too hard to construct any polygon with less than $7$ sides.

Sometimes the entire problem is reduced significantly or just solved by taking the correct cross section.

\begin{exam}
Inside a cone of radius $5$ and height $12$ there is a sphere inscribed. What is its radius?
\end{exam}
\begin{sol}
Here is a walkthrough of the solution.
\begin{enumerate}
    \item Take a cross section through the apex of the cone perpendicular to the base.
    
    \item Now you have a triangle and its incircle. Finish with $[ABC]=rs.$
\end{enumerate}
\end{sol}

\begin{exam}[AMC 10A 2019/21]
A sphere with center $O$ has radius 6. A triangle with sides of length $15$, $15$, and $24$ is situated in space so that each of its sides are tangent to the sphere. What is the distance between $O$ and the plane determined by the triangle?
\end{exam}
\begin{sol}
Here is a walkthrough of the solution.
\begin{enumerate}
    \item This is not actually a 3D geometry problem.

    \item Take a cross section of the sphere with the triangle.
    
    \item Use $[ABC]=rs$ to figure out the radius of the cross-section.
    
    \item Finish with the Pythagorean Theorem.
\end{enumerate}
\end{sol}

\section{Miscellaneous Configurations}
Here are some examples of miscellaneous techniques that can help solve 3D geometry problems.

\subsection{Pythagorean Theorem}
The Pythagorean Theorem still holds in 3 dimensions, and can be generalized to a 3 dimensional version by applying the two dimensional version twice.

\begin{exam}
Consider unit cube $ABCDEFGH,$ where $ABCD$ and $EFGH$ are opposite faces and $AG,BH,CE,DF$ are space diagonals. Find the area of triangle $AFH.$
\end{exam}

\begin{sol}
Note that $AF=FH=HA=\sqrt{2}$ by the Pythagorean Theorem, so the area is $\frac{(\sqrt{2})^2\sqrt{3}}{4}=\frac{\sqrt{3}}{2}.$
\end{sol}

\subsection{Tangent Spheres}
For problems with tangent spheres, remember the following fact.
\begin{fact}[Tangency Point is Collinear with Centers]
If two spheres with centers $O_1,O_2$ are tangent at $T,$ then $O_1,O_2,T$ are collinear.
\end{fact}
This implies the following corollary.
\begin{fact}[Distance Between Centers]
Say two spheres $\Gamma_1,\Gamma_2$ with centers $O_1,O_2$ and radii $r_1,r_2$ are tangent at $T.$ Then
\[\begin{cases}
O_1O_2=r_1+r_2 \text{ if } \Gamma_2 \text{ is externally tangent to } \Gamma_1 \\
O_1O_2=r_1-r_2 \text{ if } \Gamma_2 \text{ is internally tangent to } \Gamma_1.
\end{cases}\]
\end{fact}
Using this in conjunction with the Pythagorean Theorem is enough to solve almost all problems with tangent spheres.
\begin{exam}[AMC 12A 2004/22]
Three mutually tangent spheres of radius $1$ rest on a horizontal plane. A sphere of radius $2$ rests on them. What is the distance from the plane to the top of the larger sphere?
\end{exam}
\begin{sol}
Let's label some points. Let the centers of the unit spheres be $O_1,O_2,O_3,$ let the center of the sphere of radius $2$ be $O,$ and let the foot of the perpendicular from $O$ to $O_1O_2O_3$ be $P.$ Note that $O_1O_2O_3$ is parallel to the horizontal plane with a distance of $1.$

Note that $\triangle O_1O_2O_3$ is equilateral with side length $2,$ and by symmetry, $P$ must be the center. Thus $PO_1=\frac{2\sqrt{3}}{3}.$ Since $OO_1=3$ by Fact $2,$ $OP=\sqrt{OO_1^2-PO_1^2}=\sqrt{3^2-(\frac{2\sqrt{3}}{3})^2}=\frac{\sqrt{69}}{3}.$

Since the distance from $P$ to the horizontal plane is $1$ and the tip of the sphere with radius $2$ is $2$ above $O,$ the answer is $1+2+\frac{\sqrt{69}}{3}=3+\frac{\sqrt{69}}{3}.$
\begin{center}
\begin{tikzpicture}
    \draw (0,0)--(1.8,1.5)--(3,0)--cycle; 
    \draw (0,2)--(1.8,3.5)--(3,2)--cycle;
    \draw [dotted] (0,0)--(0,2);
    \draw [dotted] (1.8,1.5)--(1.8,3.5);
    \draw [dotted] (3,0)--(3,2);
    \draw [dotted] (0,2)--(1.5,2.5)--(1.5,5);
    \draw (1.5,5)--(0,2);
    \node at (0,2) [anchor=east] {$O_1$};
    \node at (2,3.4) [anchor=south] {$O_2$};
    \node at (3,1.95) [anchor=west] {$O_3$};
    \node at (1.4,2.5) [anchor=west] {$P$};
    \node at (1.6,5) [anchor=south] {$O$};
    \filldraw (0,2) circle (1pt);
    \filldraw (1.8,3.5) circle (1pt);
    \filldraw (3,2) circle (1pt);
    \filldraw (1.5,2.5) circle (1pt);
    \filldraw (1.5,5) circle (1pt);
\end{tikzpicture}
\end{center}
\end{sol}
Notice that the entire solution was essentially just using the correct setup and the Pythagorean Theorem.

\subsection{Unfolding}
These are problems where you take the shortest path from one point to another on the surface of a solid.

\begin{exam}
Consider a $10\times 10\times 7$ rectangular prism with $A$ as the center of a $10\times 10$ squares and $B$ as a vertex of the opposite $10\times 10$ square. If an ant crawls along the surface of the prism from $A$ to $B,$ what is the length of the shortest path he could take?
\end{exam}

\begin{sol}
Unfold the square and lay it flat with the rectangle containing $B.$ Since the shortest distance between two points is a line, $AB=\sqrt{AP^2+BP^2}=\sqrt{(\frac{10}{2})^2+(\frac{10}{2}+7)^2}=13,$ where $P$ is the foot of the altitude from $A$ to the side of the square whose extension passes through $V.$
\begin{center}
\begin{tikzpicture}[scale=0.2]
\draw (0,0)--(10,0)--(10,17)--(0,17)--cycle;
\draw (0,10)--(10,10);
\draw (10,5)--(5,5)--(10,17);
\filldraw (5,5) circle (3pt) node[anchor=south east] {$A$};
\filldraw (10,5) circle (3pt) node[anchor=south west] {$P$};
\filldraw (10,17) circle (3pt) node[anchor=south west] {$B$};
\end{tikzpicture}
\end{center}
\end{sol}

\section{Summary}

\subsection{Theory}

\begin{enumerate}
\item Prisms and Cylinders

\begin{itemize}
\Item The volume is $Bh.$

\Item The surface area of a cylinder is $2\pi r^2+2\pi rh.$

\Item The volume of a cube is $x^3.$

\Item The surface area of a cube is $6x^2.$

\Item The volume of a right rectangular prism is $lwh.$

\Item The surface area of a right rectangular prism is $2(lw+wh+hl).$
\end{itemize}
\item Pyramids and Cones
\begin{itemize}
\Item The volume is $\frac{Bh}{3}.$

\Item The surface area of a cylinder is 
\end{itemize}
\end{enumerate}

\subsection{Tips and Tricks}
\begin{enumerate}
\item Classifications
\begin{itemize}
\Item Think of prisms and cylinders as one group of solids, pyramids and cones as another, and spheres as their own thing.
\Item The design of this chapter explicitly reflects this.
\end{itemize}
\item Techniques
\begin{itemize}
\Item Think in cross-sections whenever possible.

\Item The Pythagorean Theorem still holds.

\Item Tangent spheres can be reduced to points.

\Item Problems where you find a path along a surface can be solved via unfolding.
\end{itemize}
\end{enumerate}

\pagebreak

\section{Exercises}

\subsection{Check-ins}

\begin{enumerate}
\item Consider a rectangular prism whose base has an area of $40$ and a height of $17.$ What is its volume?

\item Consider a cylinder with diameter $10$ and height $7.$ What is its volume?

\item The net of a 3D figure is composed of 6 congruent squares and has a total area of 216 square inches. When the shape is folded to form a cube, how cubic inches are in its volume?
\begin{center}
    \begin{asy}
    size(5cm);
    draw((0,0)--(4,0)--(4,1)--(0,1)--cycle);
    draw((1,-1)--(2,-1)--(2,2)--(1,2)--cycle);
    draw((3,0)--(3,1));
    \end{asy}
\end{center}

\item The numerical surface area and volume of a sphere are the same. What is the radius of this sphere?

\item Consider two right cylinders $P$ and $Q$ with the same volume. Cylinder $P$ has a radius $30\%$ longer than Cylinder $Q.$ What percent larger is the height of Cylinder $Q$ than that of Cylinder $P?$

\item A right rectangular prism with a volume of $32000$ and a base width of $8$ and a base length of $10.$ When the prism is cut by a plane parallel and equidistant to both bases, what is the combined surface area of the two remaining figures?

\item (AHSME 1996/9) Triangle $PAB$ and square $ABCD$ are in perpendicular planes. Given that $PA = 3, PB = 4$ and $AB = 5$, what is $PD?$

\begin{center}
    \begin{asy}
size(5cm);
    import olympiad;
    real r=sqrt(2)/2;
draw(origin--(8,0)--(8,-1)--(0,-1)--cycle);
draw(origin--(8,0)--(8+r, r)--(r,r)--cycle);
filldraw(origin--(-6*r, -6*r)--(8-6*r, -6*r)--(8, 0)--cycle, white, black);
filldraw(origin--(8,0)--(8,6)--(0,6)--cycle, white, black);
pair A=(6,0), B=(2,0), C=(2,4), D=(6,4), P=B+1*dir(-65);
draw(A--P--B--C--D--cycle);
dot(A^^B^^C^^D^^P);
label("$A$", A, dir((4,2)--A));
label("$B$", B, dir((4,2)--B));
label("$C$", C, dir((4,2)--C));
label("$D$", D, dir((4,2)--D));
label("$P$", P, dir((4,2)--P));
    \end{asy}
\end{center}

\item One dimension of a cube is increased by $1$, another is decreased by $1$, and the third is left unchanged. The volume of the new rectangular solid is $5$ less than that of the cube. What was the volume of the cube?

\item What is the volume of a cube whose surface area is twice that of a cube with volume 1?
\end{enumerate}

\subsection{Problems}

\begin{enumerate}
\item Consider rectangular prism $ABCDEFGH$ with dimensions $1 \times 1 \times \sqrt{3}.$ Let $AE,$ $BF,$ $CG,$ and $DH$ be perpendicular to planes $ABCD$ and $EFGH,$ and let $AE = BF = CG = DH = 1.$ Furthermore, let $AB = 1$ and $BC = \sqrt{3}.$ Find $\angle AGC.$

\item (AMC 12B 2008/18) On a sphere with a radius of $2$ units, the points $A$ and $B$ are $2$ units away from each other. Compute the distance from the center of the sphere to the line segment $AB.$

\item A parallelelepiped has coordinates $(0,0,0),(2,0,0),(1,\sqrt{3},0),(3,\sqrt{3},0)$ for one face and coordinates $(1,0,2),(3,0,2),(2,\sqrt{3},2),(4,\sqrt{3},2)$ for the opposite face. Find its surface area.

\item (AMC 12A 2005/22) A rectangular box $P$ is inscribed in a sphere of radius $r$. The surface area of $P$ is 384, and the sum of the lengths of its 12 edges is 112. What is $r$?

\item (AMC 12B 2005/16) Eight spheres of radius 1, one per octant, are each tangent to the coordinate planes. What is the radius of the smallest sphere, centered at the origin, that contains these eight spheres.

\item Consider two externally spheres $\Gamma_1$ and $\Gamma_2$ with radii $12,r,$ and consider cylinder $\Omega$ with radius $16$ and height $25.$ If $\Gamma_1$ is tangent to a base and the circumference of $\Omega$ and $\Gamma_2$ is tangent to the opposite base and the circumference of $\Omega,$ find $r.$

\item (AMC 10B 2018/10) In the rectangular parallelepiped shown, $AB$ = $3$, $BC$ = $1$, and $CG$ = $2$. Point $M$ is the midpoint of $\overline{FG}$. What is the volume of the rectangular pyramid with base $BCHE$ and apex $M$?
\begin{center}
    \begin{asy}
    import olympiad;
    size(6cm);
pair A = origin;
pair B = (4.75,0);
pair E1=(0,3);
pair F = (4.75,3);
pair G = (5.95,4.2);
pair C = (5.95,1.2);
pair D = (1.2,1.2);
pair H= (1.2,4.2);
pair M = ((4.75+5.95)/2,3.6);
draw(E1--M--H--E1--A--B--E1--F--B--M--C--G--H);
draw(B--C);
draw(F--G);
draw(A--D--H--C--D,dashed);
label("$A$",A,SW);
label("$B$",B,SE);
label("$C$",C,E);
label("$D$",D,W);
label("$E$",E1,W);
label("$F$",F,SW);
label("$G$",G,NE);
label("$H$",H,NW);
label("$M$",M,N);
dot(A);
dot(B);
dot(E1);
dot(F);
dot(G);
dot(C);
dot(D);
dot(H);
dot(M);
label("3",A/2+B/2,S);
label("2",C/2+G/2,E);
label("1",C/2+B/2,SE);
    \end{asy}
\end{center}

\item (AIME I 2020/6) A flat board has a circular hole with radius $1$ and a circular hole with radius $2$ such that the distance between the centers of the two holes is $7.$ Two spheres with equal radii sit in the two holes such that the spheres are tangent to each other. The square of the radius of the spheres is $\tfrac{m}{n},$ where $m$ and $n$ are relatively prime positive integers. Find $m+n.$

\item (AIME 1984/9) In tetrahedron $ABCD$, edge $AB$ has length 3 cm. The area of face $ABC$ is $15\mbox{cm}^2$ and the area of face $ABD$ is $12 \mbox { cm}^2$. These two faces meet each other at a $30^\circ$ angle. Find the volume of the tetrahedron in $\mbox{cm}^3$.

\item (AHSME 1996/28) On a $4\times 4\times 3$ rectangular parallelepiped, vertices $A$, $B$, and $C$ are adjacent to vertex $D$. Find the distance from $D$ to plane $ABC$.
\end{enumerate}

\subsection{Challenges}

\begin{enumerate}
\item (AIME I 2011/12) A sphere is inscribed in the tetrahedron whose vertices are $A = (6,0,0), B = (0,4,0), C = (0,0,2),$ and $D = (0,0,0).$ The radius of the sphere is $m/n,$ where $m$ and $n$ are relatively prime positive integers. Find $m + n.$

\item (AMC 10A 2013/22) Six spheres of radius $1$ are positioned so that their centers are at the vertices of a regular hexagon of side length $2$. The six spheres are internally tangent to a larger sphere whose center is the center of the hexagon. An eighth sphere is externally tangent to the six smaller spheres and internally tangent to the larger sphere. What is the radius of this eighth sphere?

\item (AMC 12B 2004/19) A truncated cone has horizontal bases with radii $18$ and $2$. A sphere is tangent to the top, bottom, and lateral surface of the truncated cone. What is the radius of the sphere?

\item (AIME II 2020/7) Two congruent right circular cones each with base radius $3$ and height $8$ have axes of symmetry that intersect at right angles at a point in the interior of the cones a distance $3$ from the base of each cone. A sphere with radius $r$ lies inside both cones. The maximum possible value for $r^2$ is $\frac mn$, where $m$ and $n$ are relatively prime positive integers. Find $m+n$.

\item (AIME I 2013/7) A rectangular box has width $12$ inches, length $16$ inches, and height $\frac{m}{n}$ inches, where $m$ and $n$ are relatively prime positive integers. Three faces of the box meet at a corner of the box. The center points of those three faces are the vertices of a triangle with an area of $30$ square inches. Find $m+n$.

\item (AIME II 2016/14) Equilateral $\triangle ABC$ has side length $600$. Points $P$ and $Q$ lie outside the plane of $\triangle ABC$ and are on opposite sides of the plane. Furthermore, $PA=PB=PC$, and $QA=QB=QC$, and the planes of $\triangle PAB$ and $\triangle QAB$ form a $120^{\circ}$ dihedral angle (the angle between the two planes). There is a point $O$ whose distance from each of $A,B,C,P,$ and $Q$ is $d$. Find $d$.
\end{enumerate}